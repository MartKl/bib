% !TEX root=./template.tex %%% ref to the main file

% new thm environments
\newtheorem{theorem}{Theorem}
\newtheorem{corollary}[theorem]{Corollary}
\newtheorem{lemma}[theorem]{Lemma} % "[theorem]" makes the lemma environment to use the same counter as the theorem environment
\newtheorem{proposition}[theorem]{Proposition}
\newtheorem{definition}[theorem]{Definition}


%%%=====================================================
%%%================  new commands  =======================
%%%====================================================
%%% ------ rm ---------
\newcommand{\e}{\ensuremath\mathrm{e}} % Euler e
\renewcommand{\i}{\ensuremath\mathrm{i}} % imaginary unit
\DeclareMathOperator{\Tr}{Tr} % trace 
\renewcommand{\Re}{\operatorname{Re}} % real part
\renewcommand{\Im}{\operatorname{Im}} % imaginary part
\DeclareMathOperator{\ran}{ran} % range
\DeclareMathOperator{\rank}{rank} % rank 
\DeclareMathOperator{\vecmap}{vec} % vectorization map
\renewcommand{\L}{\operatorname{L}} % linear operators
\DeclareMathOperator{\sign}{sign} % the sign map

\DeclareMathOperator{\U}{U}%unitary group U(n) 
\renewcommand{\O}{\operatorname{O}}%orthogonal group U(n)
\DeclareMathOperator{\SU}{SU}%special unitary group SU(n)

%%% ------ mathbb --------
\newcommand{\CC}{\mathbb{C}}% complex numbers
\newcommand{\RR}{\mathbb{R}}% reals
\newcommand{\KK}{\mathbb{K}}% for \RR or \CC
\newcommand{\QQ}{\mathbb{Q}}
\newcommand{\ZZ}{\mathbb{Z}}
\newcommand{\NN}{\mathbb{N}}
\newcommand{\1}{\mathds{1}} % identity operator
\newcommand{\EE}{\mathbb{E}} % expectation value
\newcommand{\PP}{\mathbb{P}} % Prob.

%%% ------ mathcal ---------
\newcommand{\mc}[1]{\mathcal{#1}}
\renewcommand{\H}{\mc{H}} % Hilbert space 


%%% ------ other ----------
\newcommand{\argdot}{{\,\cdot\,}} % for a dot as an argument
\renewcommand{\vec}[1]{\mathbf{#1}} % for vectors
\newcommand{\ad}{\dagger} % $A^\ad$ is the adjoint of $A$

\DeclareMathOperator{\LandauO}{\mathrm{O}} % Landau big-O notation
\DeclareMathOperator{\LandauOmega}{\Omega}
\DeclareMathOperator{\tLandauO}{\tilde{\mathrm{O}}}
\DeclareMathOperator{\tLandauOmega}{\tilde{\Omega}}
\DeclareMathOperator{\LandauTheta}{\Theta}
\DeclareMathOperator{\tLandauTheta}{\tilde{\Theta}}


%%% ------ norms, inner product ----------
% requires: \usepackage{ifthen}
\DeclarePairedDelimiterX{\abs}[1]{\lvert}{\rvert}{%
  \ifblank{#1}{\,\cdot\,}{#1}
}   % absolute value

\DeclarePairedDelimiterX\norm[1]\lVert\rVert{%
  \ifblank{#1}{\,\cdot\,}{#1}
}   % norm


\newcommand{\lpnorm}[2][p]{\norm{#2}_{\ell_{#1}}}   %Lp norm - standard size
\newcommand{\lpnorma}[2][p]{\norm*{#2}_{\ell_{#1}}}   %Lp norm - automatic size
\newcommand{\lpnormb}[2][p]{\norm[\big]{#2}_{\ell_{#1}}}
\newcommand{\lpnormB}[2][p]{\norm[\Big]{#2}_{\ell_{#1}}}
%
\newcommand{\pnorma}[2][p]{\norm*{#2}_{#1}} %p norm - standard size
\newcommand{\pnorm}[2][p]{\norm{#2}_{#1}} %p norm - automatic size
\newcommand{\pnormb}[2][p]{\norm[\big]{#2}_{#1}}
\newcommand{\pnormB}[2][p]{\norm[\Big]{#2}_{#1}}
%
\DeclarePairedDelimiterX{\iiiNorm}[1]{\lvert}{\rvert}{%
  \delimsize\lvert\delimsize\lvert#1\delimsize\rvert\delimsize\rvert%
}

% \DeclarePairedDelimiterXPP{〈cmd〉}[〈num args〉]{〈pre code〉}{〈left_delim〉}{〈right_delim〉}{〈post code〉}{〈body〉}
% interpreted as 
% {〈pre code〉} {〈left_delim〉} {〈body〉} {〈right_delim〉} {〈post code〉}
\DeclarePairedDelimiterXPP\snorm[1]{}\lVert\rVert{_\infty}{\ifblank{#1}{\,\cdot\,}{#1}}   %spectral norm  =  (2->2)-norm
\newcommand{\snormb}[1]{\snorm[\big]{#1}}
\newcommand{\snormB}[1]{\snorm[\Big]{#1}}
%
\DeclarePairedDelimiterXPP\twonorm[1]{}\lVert\rVert{_2}{\ifblank{#1}{\,\cdot\,}{#1}}   % 2-norm
\newcommand{\twonormb}[1]{\twonorm[\big]{#1}}

\DeclarePairedDelimiterXPP\trnorm[1]{}\lVert\rVert{_1}{\ifblank{#1}{\,\cdot\,}{#1}}   % trace norm
\newcommand{\trnormb}[1]{\trnorm[\big]{#1}}

\DeclarePairedDelimiterXPP\fnorm[1]{}\lVert\rVert{_{\fro}}{\ifblank{#1}{\,\cdot\,}{#1}}   % Fro-norm
\newcommand{\fnormn}[1]{\fnorm*{#1}}
\newcommand{\fnormb}[1]{\fnorm[\big]{#1}}

\DeclarePairedDelimiterXPP\dnorm[1]{}\lVert\rVert{_\diamond}{\ifblank{#1}{\,\cdot\,}{#1}}   % diamond norm
\newcommand{\dnormb}[1]{\dnorm[\big]{#1}}

\DeclarePairedDelimiterXPP\cbnorm[1]{}\lVert\rVert{_\mathrm{cb}}{\ifblank{#1}{\,\cdot\,}{#1}}   % CB-norm
\DeclarePairedDelimiterXPP\onenorm[1]{}\lVert\rVert{_{1\rightarrow 1}}{\ifblank{#1}{\,\cdot\,}{#1}}   % (1->1)-norm
\DeclarePairedDelimiterXPP\ddnorm[1]{}\lVert\rVert{_{\diamond\rightarrow \diamond}}{\ifblank{#1}{\,\cdot\,}{#1}}   % (\diamond->\diamond)-norm
\DeclarePairedDelimiterXPP\ssnorm[1]{}\lVert\rVert{_{\infty\rightarrow\infty}}{\ifblank{#1}{\,\cdot\,}{#1}}   % (\infty->\infty)-norm

% set
% just to make sure it exists
\providecommand\given{}
% can be useful to refer to this outside \Set
\newcommand\SetSymbol[1][]{%
  \nonscript\:#1\vert
  \allowbreak
  \nonscript\:
  \mathopen{}}
\DeclarePairedDelimiterX\Set[1]\{\}{%
  \renewcommand\given{\SetSymbol[\delimsize]}
  #1
}

%inner product
\DeclarePairedDelimiterX\innerp[2]{\langle}{\rangle}{%
  \ifblank{#1}{\,\cdot\,}{#1} , \ifblank{#2}{\,\cdot\,}{#2}%
}

% ket-bra-notation
\DeclarePairedDelimiter{\bra}{\langle}{\vert}
\DeclarePairedDelimiter{\ket}{\vert}{\rangle}
\newcommand{\ketb}[1]{\ket[\big]{#1}}
\newcommand{\brab}[1]{\bra[\big]{#1}}

\DeclarePairedDelimiterX\braket[2]{\langle}{\rangle}%
  {#1\kern0.15ex\delimsize\vert\kern0.15ex\mathopen{}#2}
\newcommand{\braketb}[2]{\braket[\big]{#1}{#2}}
\newcommand{\braketB}[2]{\braket[\Big]{#1}{#2}}

\DeclarePairedDelimiterX\ketbra[2]{\vert}{\vert}%
  {#1\kern0.15ex\delimsize\rangle\delimsize\langle\kern0.15ex\mathopen{}#2}

\DeclarePairedDelimiterX\sandwich[3]{\langle}{\rangle}%
  {#1\,\delimsize\vert\kern0.15ex\mathopen{}#2\kern0.15ex\delimsize\vert\kern0.15ex\mathopen{}#3}

% ket-bra-notation with round deliminators
\DeclarePairedDelimiter{\obra}{(}{\vert}
\DeclarePairedDelimiter{\oket}{\vert}{)}

\DeclarePairedDelimiterX\obraket[2]{(}{)}%
  {#1\kern0.15ex\delimsize\vert\kern0.15ex\mathopen{}#2}

\DeclarePairedDelimiterX\oketbra[2]{\vert}{\vert}%
  {#1\kern0.15ex\delimsize)\delimsize(\kern0.15ex\mathopen{}#2}

\DeclarePairedDelimiterX\osandwich[3]{(}{)}%
  {#1\,\delimsize\vert\kern0.15ex\mathopen{}#2\kern0.15ex\delimsize\vert\kern0.15ex\mathopen{}#3}
\newcommand{\osandwichb}[3]{\osandwich[\big]{#1}{#2}{#3}}

% %------- smaller space in,e.g., $\exp\left(x\right)$ -> $\exp\paren{x}$
\DeclarePairedDelimiter\paren{(}{)}

% verbatim font
\newcommand{\verbat}[1]{{\ttfamily #1}}
% --------------- this project -----------------


% ... more ...

